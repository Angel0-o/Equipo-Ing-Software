%---------------------------------------------------------
\section{Introducción}

\begin{description}

\item Como parte de la Unidad de Aprendizaje de Ingeniería de Software, se propone realizar un sistema que permita automatizar la mayor cantidad de las actividades del Club de fútbol de la Escuela Superior de Cómputo, las cuales son llevadas principalmente por el Coordinador éste, lo cual produce el riesgo de que toda la administración y funcionamiento dependa de la disposición del mismo.\\
	
Así pues, se presenta la documentación de un sistema que brindará a sus usuarios una rapidez y eficiencia mayor a la actual en la organización de los torneos de fútbol rápido y de asociación; contemplando el registro de equipos y jugadores, la publicación de convocatorias, calendarios de juego y estadísticas de cada torneo.\\

Entre las métricas que se plantea que cubrirá el sistema, se considera que sea de gran usabilidad, orientado a usuarios poco familiarizados con las interfaces de usuario en dispositivos móviles o computadoras, además de ser flexible y escalable, de tal forma que a futuro se puedan agregar módulos para contabilizar las horas de entrenamiento al formar parte del club, así como generar las respectivas constancias de la Unidad de Aprendizaje Electiva.

\end{description}

%---------------------------------------------------------
